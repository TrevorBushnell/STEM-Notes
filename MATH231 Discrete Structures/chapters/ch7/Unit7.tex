\section{Recursive Relations and Recursive Definitions}

\begin{problem}
    Find the first four terms of the sequence if:

    \begin{itemize}
        \item $f(n+1) = 3f(n);\:\:\:f(0)=2$
        \begin{itemize}
            \item $f(1)=3f(0)=3(2)=6$
            \item $f(2)=3f(1)=3(6)=18$
            \item $f(3)=3f(2)=3(18)=54$
            \item $f(4) = 3f(3)= 3(54) = 162$
        \end{itemize}
    \end{itemize}
\end{problem}

\begin{problem}
    A ternary is a string consisting of 0s, 1s, and 2s.

    \begin{itemize}
        \item Find a recurrence relation for the number of ternary strings of length $n$ that contain a pair of consecutive 2s.
        \begin{itemize}
            \item $a_n = 2a_{n-1} + 2a_{n-2} + 3^{n-2}$
        \end{itemize}

        \item What are the initial condition(s)?
        \begin{itemize}
            \item $a_2 = 2$
            \begin{itemize}
                \item 22
            \end{itemize}
            \item $a_3 = 5$
            \begin{itemize}
                \item 220
                \item 221
                \item 222
                \item 022
                \item 122
            \end{itemize}
            \item $a_4 = 2a_3 + 2a_2 + 3^2 = 21$ 
        \end{itemize}

        \item Determine the number of ternary strings of length 5 that contain a pair of consecutive 2s.
        \begin{itemize}
            \item $a_5 = 2a_4 + 2a_3 + 3^3 = 2(21)+2(5)+3^3 = 79$
        \end{itemize}
    \end{itemize}
\end{problem}



\section{Solving Linear Recurrence Relations}

\begin{itemize}
    \item a sequence $\{a_n\}$ is a \textbf{solution} to a recurrence relation if its terms satisfy the recurrence relation
\end{itemize}


\begin{problem}
    Which of the following sequences are solutions of the recurrence relation $a_n = 8a_{n-1} - 16a_{n-2}$?

    \begin{itemize}
        \item $a_n = 0$ $\implies$ [ADD WORK LATER] 
        \item $a_n = 2^n$ \textbf{is NOT a solution}
        \[
        \begin{aligned}
            &8\cdot 2^{n-1}-16\cdot 2^{n-2}\\
            &2^3\cdot 2^{n-1}-2^4\cdot 2^{n-2}\\
            &2^{n+2} - 2^{n+2} = 0
        \end{aligned}    
        \]
        \item $a_n = 3n4^n$ \textbf{IS a solution}
        \[
        \begin{aligned}
            &8\cdot 3(n-1)\cdot 4^{n-1} - 16 \cdot 3(n-2) \cdot 4^{n-2}\\
            &2\cdot 4\cdot 3(n-1)4^{n-1} - 4\cdot 4 \cdot 3 (n-2) 4^{n-2}\\
            &6(n-1)4^n - 3(n-2)4^n\\
            &4^n(6n-6-3n+6)\\
            &3n4^n
        \end{aligned}    
        \]
    \end{itemize}
\end{problem}

\subsection{Solving Recurrence Relations}


If the recurrence relation only depends on \textbf{one term}, use the following process:
\begin{itemize}
    \item To solve a recurrence relation $a_n=ca_{n-1}$, look at the root of the equation $r-c=0$
    \item the root is $r_1 = c$
    \item genereal form of the solution then is $a_n=\alpha (r_1)^n$ such that $n \ge 0$
    \item Use the initial condition to solve for $\alpha$
\end{itemize}

If the recurrence relation depends on \textbf{two terms}, use the following process:

\begin{itemize}
    \item To solve a recurrence relation $a_n=c_1a_{n-1} + c_2a_{n-2}$, look at the root of the equation $r^2-c_1r-c_2=0$
    \item the roots can be found using the quadratic fomula or by factoring
    \item genereal form of the solution then is $a_n=\alpha (r_1)^n + \beta (r_2)^n$ such that $n \ge 0$
    \item Use the initial conditions to solve for $\alpha$ and $\beta$
\end{itemize}


\begin{problem}
    Solve $a_n = 5a_{n-1}\:\:\:(n\ge 1),\:\: a_0 = -2$.
\end{problem}

\begin{problem}
    Solve $a_n = a_{n-1}+6a_{n-2}\:\:\:(n\ge 2),\:\: a_0 = 3 \:\&\: a_1 = 6$.
\end{problem}