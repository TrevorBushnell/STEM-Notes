\section{Graphs and Terminology}

\begin{definition}[Graphs]{def8.1.1:label}
    \begin{itemize}
        \item A \textbf{graph} $G=(V,E)$ consists of a nonempty set of vertices and a set $E$ of edges. Each edge has either one or two vertices associated with it, called its \textit{endpoints}. 
        \item A \textbf{loop} is an edge between a vertex and itself. 
        \item A \textbf{multiple edge} is one in which the same pair of vertices is assigned to the more than one edge.  
        \item an \textbf{unordered edge} is denoted by an unordered pair of vertices
        \item a \textbf{directed edge} is denoted by an ordered pair of vertices
    \end{itemize}
\end{definition}


\subsection{Types of Graphs}

\begin{itemize}
    \item \textbf{UNDIRECTED GRAPHS} have undirected edges. The \textit{degree} of a vertex $v$ in an undirected graph is the number of edges which have $v$ as an endpoint (loops count twice)
    \item \textbf{SIMPLE GRAPHS} are undirected graphs with no loops or multiple edges
\end{itemize}

\textbf{INCLUDE PROBLEM HERE}

\begin{itemize}
    \item \textbf{MULTIGRAPHS} are undirected graphs which allow multiple edges but no loops
\end{itemize}

\textbf{INCLUDE PROBLEM HERE}

\begin{itemize}
    \item \textbf{PSEUDOGRAPHS} are undirected graphs that allow both multiple edges and loops
\end{itemize}

\textbf{INCLUDE PROBLEM HERE}

\begin{itemize}
    \item \textbf{DIRECTED GRAPHS} have \textit{directed edges}. The \textit{in-degree} of a vertex in a directed graph, denoted by $\deg^-(v)$ is the number of edges that terminate at $v$. The \textit{out-degree} of $v$, denoted by $\deg^+(v)$ is the number of edges starting at $v$
    \item \textbf{SIMPLE DIRECTED GRAPHS} are directed graphs with no loops or multiple edges
\end{itemize}

\textbf{INCLUDE PROBLEM HERE}

\begin{itemize}
    \item \textbf{DIRECTED MULTIGRAPHS} are directed graphs which allow both loops and multiple edges
\end{itemize}


\begin{theorem}[The Handshake Theorem]{th8.1.2:label}
    \begin{itemize}
        \item For \textit{undirected graphs}, $2|E| = \sum_{v\in V} \deg(v)$
        \item For \textit{directed graphs}, $|E| = \sum_{v\in V} \deg^-(v) = \sum_{v\in V} \deg^+(v)$
    \end{itemize}
\end{theorem}


\subsection{Families of Graphs}

\begin{itemize}
    \item The \textbf{COMPLETE GRAPH} on $n$ vertices, denoted $K_n$, consists of $n$ vertices and edges between every pair of distinct vertices.
    \item In general, $K_n$ has...
    \begin{itemize}
        \item $|V_n| = n$
        \item $|E_n| = |E_{n-1}| + (n-1) \sum_{i=0}^{n-1}i = \frac{n(n-1)}{2}$
    \end{itemize} 
    \item CIRCLE GRAPH
    \item WHEEL RGAPH
    \item \textbf{$n$-CUBES (HYPERCUBES)} denoted $Q_n(n\ge 1)$ consists of a set of vertices which correspond to all bit strnig of length $n$ along with a set of edges connecting vertices exactly when the corresponding bit strings differ in exactly one bit.
    \item In general, $Q_n$ has\dots
    \begin{itemize}
        \item $|V_n| = $
        \item $|E_n| = $
    \end{itemize}
\end{itemize}



\subsection{Even More Terminology}

\begin{itemize}
    \item Two vertices are \textbf{adjacent} 
\end{itemize}