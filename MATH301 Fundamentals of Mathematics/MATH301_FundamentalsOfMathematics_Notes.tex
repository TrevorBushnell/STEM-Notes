
\documentclass{package/notes}
\usepackage[english]{babel}
\usepackage{amssymb,amsmath,amsfonts}  %%% for maths
%%%%%%%%%%%%%%%%%%%%%%%%%%%%%%%%%%%%%
\usepackage{package/color-env}
\usepackage{lipsum}

\newcommand{\Z}{\mathbb{Z}}
\newcommand{\N}{\mathbb{N}}
\newcommand{\R}{\mathbb{R}}
\newcommand{\Q}{\mathbb{Q}}
\newcommand{\C}{\mathbb{C}}
\newcommand{\lcm}{\text{lcm}}
\renewcommand\qedsymbol{$\blacksquare$}
%%%%%%%%%%%%%%%%%%%%%%%%%%%%%%%%%%%%%

\begin{document}

	\begin{titlepage} % Suppresses headers and footers on the title page
		
		\centering % Centre everything on the title page
		
		\scshape % Use small caps for all text on the title page
		
		\vspace*{\baselineskip} % White space at the top of the page
		
		%------------------------------------------------
		%	Title
		%------------------------------------------------
		
		\rule{\textwidth}{1.6pt}\vspace*{-\baselineskip}\vspace*{2pt} % Thick horizontal rule
		\rule{\textwidth}{0.4pt} % Thin horizontal rule
		
		\vspace{0.75\baselineskip} % Whitespace above the title
		
		{\huge MATH301: Fundamentals of Mathematics Notes\\} % Title
		
		\vspace{0.75\baselineskip} % Whitespace below the title
		
		\rule{\textwidth}{0.4pt}\vspace*{-\baselineskip}\vspace{3.2pt} % Thin horizontal rule
		\rule{\textwidth}{1.6pt} % Thick horizontal rule
		
		\vspace{2\baselineskip} % Whitespace after the title block
		
		%------------------------------------------------
		%	Subtitle
		%------------------------------------------------
		 
		
		\vspace*{3\baselineskip} % Whitespace under the subtitle
		
		
		
		\vspace{0.5\baselineskip} 
		
		\vspace{0.5\baselineskip} 
		
		\vfill 
		
		%------------------------------------------------
		% Author
		%------------------------------------------------
		
		
		\vspace{0.3\baselineskip} 
		
		
		{\large Edited by\\  Trevor Bushnell} 
		
	\end{titlepage}
	\tableofcontents
%\newpage

\chapter*{Introduction}

This document aims to highlight the important content of the MATH301 course in traditional notes format. These notes are completely open-source, which means anyone is allowed to use these notes for their own personal benefit without having to seek permission from myself. \newline

Due to the open-source nature of these notes, anyone is allowed to contribute to improving these notes as they see fit. Since I am using \LaTeX to write these notes and I am using GitHub to distribute these notes easily, you must request all changes through the repository website on GitHub, which you can find \textbf{here}. If you are interested in contributing to these notes, then there are a few ways that you can do so:\newline

\begin{enumerate}
	\item \textbf{Open and submit an issue on my GitHub repository:} I write all my notes in \LaTeX, which is a typesetting language that is really helpful when it comes to typing and rendering math equations quickly and easily. If you do not know how to write \LaTeX code but are still interested in making a change to the notes, you can open an issue by going to the MathNotes repo on GitHub, and clicking on the button labeled "New Issue." From there, you can type out the change that you wish to see in the notes. It would be helpful if you would indicate what course you would like to see changed so that I can understand what you are referring to. I will then update the code to include your issue so that you don't have to worry about writing the code yourself.
	\item \textbf{Create and submit a pull request:} If you know how to write LaTeX code and you understand how GitHub works, you can submit a pull request where you can write the code that you want to change yourself. I will then review the code and either submit the code to be incorporated into the notes OR provide some comments on your code if I wish for something to be different. 
\end{enumerate}

Thank you so much for using these notes. I hope that the information is provided in such a way that it can help you when reviewing content for you AP test/class exam and just in general when it comes to learning the content for the course. Happy studying!

\chapter{Set Theory}
\section{Propositional Logic}

Logic can most simply be broken down into statements. If a statement has a definitive \it{TRUE} or \it{FALSE} value, then the statement is called a \bf{proposition}. \\


\subsection{Logical Operators}

If we suppose that $p,q,r,s$ can be any propositional statement that could be either true or false, then we can build \it{truth tables} though the use of various \bf{logical operators} which are explained below:

\begin{itemize}
    \item \bf{NEGATION $\neg$:} Pronounced "not $p$"; takes the opposite value of whatever $p$ is. This is a \it{unitary operator} because it acts on only one proposition
    \item \bf{CONJUNCTION $\wedge$:} Pronounced "$p$ and $q$; this is true when both $p$ and $q$ are true and false in every other instance
    \item \bf{DISJUNCTION $\vee$:} Prononced "$p$ or $q$"; this is true when either $p$ or $q$ are true and false in every other instance (an easier way to remember this is that the disjunction is false when both $p$ and $q$ are false and true in every other instance)
    \item \bf{EXCLUSIVE OR $\oplus$:} Pronounced "$p$ or $q$, but not both"; this is true when $p$ and $q$ have opposite vales and false when they have the same value
    \item \bf{CONDITIONAL STATEMENT $\rightarrow$:} Pronounced "If $p$ then $q$"; this statement is false when $p$ is true and $q$ is false and true in all other cases 
    \item \bf{BICONDITIONAL STATEMENTS $\leftrightarrow$:} Pronounced "$p$ if and only if $q$", this is true when $p$ and $q$ have the same truth values and false in all other cases
\end{itemize}

The following are the truth tables for each of the logical operators described above:\\



\subsection{Types of Conditional Statements}

A conditional statement can take on many different forms. If $p \rightarrow q$ is the original conditional statement, then the following are variations of this statement:

\begin{itemize}
    \item \bf{CONVERSE:} $q \rightarrow p$
    \item \bf{CONTRAPOSITIVE:} $\neg q \rightarrow \neg p$
    \item \bf{INVERSE:} $\neg p \rightarrow \neg q$
\end{itemize}

\bf{INSERT A PROBLEM ABOUT THIS HERE}

\bf{INSERT TRUTH TABLES FOR ALL 4 OF THESE HERE}


\subsection{Bitwise Operations}

If we say that TRUE = 1 and FALSE = 0, then we can use bit strings to represent sets of propositions. To do bitwise operation on these, simply perform the operation on each corresponding bit in the same position.\\

\bf{INSERT PROBLEM WITH BIWISE OPERATION HERE}



\section{Propositional Equivalences}

\begin{definition}{def1.2.1:label}
    A compound proposition $r$ that is always true regardless of the truth values of the individual propositions is called a \bf{tautology}, written as $r \equiv T$.\\

    A compound proposition $r$ that is always false regardless of the truth values of the individual propositions is called a \bf{contradiction}, written as $r \equiv T$.\\

    A compound proposition $r$ that is neither a tautology nor a contradiction is called a bf{contingency}.
\end{definition}

\begin{definition}{def1.2.2:label}
    Two compound propositional statements $r$ and $s$ that have identical truth columns are called \bf{logically equivalent} and are written as $r \equiv s$.
\end{definition}


\begin{center}
    \includegraphics[width=0.75\textwidth]{chapters/ch1/images/fig1.2.1.PNG}
\end{center}



\section{Predicates and Quantifiers}

Before we dive into this section, it is important to be on the same page in terms of some common notation:

\begin{itemize}
    \item The natural numbers: $\N = \{0,1,2,3,4,\dots\}$
    \item The integers: $\Z = \{\dots -2, -1, 0, 1, 2, \dots\}$
    \item the positive integers: $\Z^+ = \{1,2,3,4,\dots\}$
    \item the rational numbers: $\Q = \{\frac{a}{b} | a, b $ are integers where $ b \ne 0\}$
    \item the real numbers: $\R = (-\infty, \infty)$
    \item the positive real numebrs: $\R^+ = (0, \infty)$
    \item the complex numbers: $\C = \{a+bi | a,b $ are real numbers$\}$
\end{itemize}


\chapter{Logic}
\input{chapters/ch2/Unit2.tex}



\chapter{Direct Proofs}
\section{Helpful Definitions for Direct Proofs}

The following are definitions that will be useful when constructing direct proofs (and any proof from this point forward). Many of these defintions have to do with even/odd numbers, divisibility, and prime numbers.

\begin{definition}[Definition of Even and Odd Numbers]{def3.1:label}
    An integer $n$ is \textbf{even} if $n=2k$ for some $k \in \Z$\\

    An integer $n$ is \textbf{odd} if $n=2k+1$ for some $k \in \Z$\\

    Two integers have the same \textbf{parity} if both integers are even/odd, otherwise they have the opposite parity.
\end{definition}

\begin{definition}[Definition of Divisibility]{def3.2:label}
    Suppose $a$ and $b$ are integers. Then $a$ \textbf{divides} $b$ (written $a|b$) if $b = ac$ for some $c\in \Z$. We would also say that $a$ is a \textbf{divisor} of $b$ and that $b$ is a \textbf{multiple} of $a$. 
\end{definition}

\begin{definition}[Prime Numbers]{def3.3:label}
    A natural number $n$ is \textbf{prime} if it has exactly \textit{two} positive divisors, 1 and $n$.\\

    A natural number $n$ is \textbf{composite} if it factors as $n=ab$ where $a,b>1$.
\end{definition}

\textbf{NOTE:} Do not use the statements $a/b$ or $a/b \in \Z$ because without careful planning of your argument, you might be making claims about rational numbers rather than integers!
\newpage

\begin{definition}[$\gcd$ and $\lcm$]{def3..4:label}
    The \textbf{greatest common divisor} of two integers $a$ and $b$ (notation: $\gcd(a,b)$) is the largest integer that divides both $a$ and $b$. \\

    The \textbf{least common multiple} of nonzero integers $a$ and $b$ (denoted $\lcm(a,b)$) is the smallest positive integer that is a multiple of both $a$ and $b$.
\end{definition}

\begin{definition}[The Dvision Algorithm]{def3.5:label}
    Given integers $a$ and $b$ with $b>0$, there exist unique integers $q$ and $r$ for which $a=qb+r$ and $0 \le r < b$.\\

    Also, any integer greater than 1 can be uniquely factored as a product of primes (prime factorization).
\end{definition}

\section{Outline for a Direct Proof}

To prove the statement $P\implies Q$, go through the following outline in your proof:

\begin{itemize}
    \item "Suppose $P$ is true"
    \item INSERT YOU MATH WORK/LOGIC/EXPLANATIONS
    \item Final step where you end up with $Q$
    \item $\therefore$ Q is true
    \item $\therefore P \implies Q$ is true
    \item \textbf{End of proof.} 
\end{itemize}
\newpage

\section{Direct Proof Examples}

\begin{proposition}{prop3.1:label}
    If $x$ is odd, then $x^2$ is odd.
\end{proposition}

\begin{proof}
    Suppose $x$ is odd. Then for some $a \in \Z$, we have $x = 2a+1$. As such, we have the following:

    $$
    \begin{aligned}
        x^2 &= (2a+1)^2\\
        &= 4a^2 + 4a + 1\\
        &= 2(2a^2+2a)+1
    \end{aligned}
    $$

    Since $a \in \Z$, then $2a^2+2a \in \Z$. Therefore, $x^2$ is odd.\\

    $\therefore$ if $x$ is odd, then $x^2$ is odd.
\end{proof}


\begin{proposition}{prop3.2:label}
    Let $a,b,c$ be integers. Then if $a|b$ and $b|c$, then $a|c$. 
\end{proposition}

\begin{proof}
    Suppose that the statements $a|b$ and $b|c$ are true. Then there is some integer $m$ such that $am = b$ and there is some integer $n$ such that $bn = c$. By substituting $am = b$ into $bn = c$, we get the following:

    $$
    \begin{aligned}
        c = (am)n\\
        c = a(mn)
    \end{aligned}
    $$

    Since $m$ and $n$ are both integers, then $mn$ must be an integer. Since $a$ times some integer equals $c$, then $a$ divides $c$.\\

    $\therefore$ if $a|b$ and $b|c$, then $a|c$.
\end{proof}

\begin{proposition}{prop3.3:label}
    If $a,b,c \in \Z^+$, then $\lcm(ca,cb) = c\lcm(a,b)$
\end{proposition}

\begin{proof}
    \textbf{INSERT PROOF HERE}
\end{proof}

\section{Proof By Cases}

Many times, an effective proof must consider all possible forms that a statement can take on. For example, if your statement in a proof is something along the lines of "$n \in \Z$", there might be different outcomes depending on whether $n$ is positive, negative, or zero. If you can effectively "cover all the bases" and prove each case individually, then you can more effectively make a conslucion about the statement as a whole.\\

When you use cases, ensure that you do the following:
\begin{itemize}
    \item clearly state what the cases are
    \item cover all the possible cases in your proof
\end{itemize}

\begin{proposition}{prop3.4.1:label}
    If $n \in \N$, then $1 + (-1)^n(2n-1)$ is a multiple of 4.
\end{proposition}

\begin{proof}
    Suppose $n \in \N$. Then $n$ is either even or odd. Consider these two cases:\\

    \textbf{Case 1- $n$ is even:} If $n$ is even, then there is some integer  $k$ such that $n = 2k$. \\

    \textbf{Case 2- $n$ is odd:} \\

    $\therefore$ if $n \in \N$, then $1 + (-1)^n(2n-1)$ is a multiple of 4. (\textit{finish this proof later})
\end{proof}










\end{document}
.