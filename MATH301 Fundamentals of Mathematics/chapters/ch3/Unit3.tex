\section{Helpful Definitions for Direct Proofs}

The following are definitions that will be useful when constructing direct proofs (and any proof from this point forward). Many of these defintions have to do with even/odd numbers, divisibility, and prime numbers.

\begin{definition}[Definition of Even and Odd Numbers]{def3.1:label}
    An integer $n$ is \textbf{even} if $n=2k$ for some $k \in \Z$\\

    An integer $n$ is \textbf{odd} if $n=2k+1$ for some $k \in \Z$\\

    Two integers have the same \textbf{parity} if both integers are even/odd, otherwise they have the opposite parity.
\end{definition}

\begin{definition}[Definition of Divisibility]{def3.2:label}
    Suppose $a$ and $b$ are integers. Then $a$ \textbf{divides} $b$ (written $a|b$) if $b = ac$ for some $c\in \Z$. We would also say that $a$ is a \textbf{divisor} of $b$ and that $b$ is a \textbf{multiple} of $a$. 
\end{definition}

\begin{definition}[Prime Numbers]{def3.3:label}
    A natural number $n$ is \textbf{prime} if it has exactly \textit{two} positive divisors, 1 and $n$. A natural number $n$ is \textbf{composite} if it factors as $n=ab$ where $a,b>1$.
\end{definition}

\textbf{NOTE:} Do not use the statements $a/b$ or $a/b \in \Z$ because without careful planning of your argument, you might be making claims about rational numbers rather than integers!
\newpage

\begin{definition}[$\gcd$ and $\lcm$]{def3..4:label}
    The \textbf{greatest common divisor} of two integers $a$ and $b$ (notation: $\gcd(a,b)$) is the largest integer that divides both $a$ and $b$. \\

    The \textbf{least common multiple} of nonzero integers $a$ and $b$ (denoted $\lcm(a,b)$) is the smallest positive integer that is a multiple of both $a$ and $b$.
\end{definition}

\begin{definition}[The Dvision Algorithm]{def3.5:label}
    Given integers $a$ and $b$ with $b>0$, there exist unique integers $q$ and $r$ for which $a=qb+r$ and $0 \le r < b$.\\

    Also, any integer greater than 1 can be uniquely factored as a product of primes (prime factorization).
\end{definition}

\section{Outline for a Direct Proof}

To prove the statement $P\implies Q$, go through the following outline in your proof:

\begin{itemize}
    \item "Suppose $P$ is true"
    \item INSERT YOU MATH WORK/LOGIC/EXPLANATIONS
    \item Final step where you end up with $Q$
    \item $\therefore$ Q is true
    \item $\therefore P \implies Q$ is true
    \item \textbf{End of proof.} 
\end{itemize}

\textbf{NOTE:} You may use any of the following statements without writing a proof:\\

For any $a,b\in\Z$:

\begin{itemize}
    \item $a+b\in\Z$
    \item $a-b\in\Z$
    \item $ab \in \Z$
\end{itemize}
\newpage

\section{Direct Proof Examples}

\begin{proposition}{prop3.1:label}
    If $x$ is odd, then $x^2$ is odd.
\end{proposition}

\begin{proof}
    Suppose $x$ is odd. Then for some $a \in \Z$, we have $x = 2a+1$. As such, we have the following:

    $$
    \begin{aligned}
        x^2 &= (2a+1)^2\\
        &= 4a^2 + 4a + 1\\
        &= 2(2a^2+2a)+1
    \end{aligned}
    $$

    Since $a \in \Z$, then $2a^2+2a \in \Z$. Therefore, $x^2$ is odd.\\

    $\therefore$ if $x$ is odd, then $x^2$ is odd.
\end{proof}


\begin{proposition}{prop3.2:label}
    Let $a,b,c$ be integers. Then if $a|b$ and $b|c$, then $a|c$. 
\end{proposition}

\begin{proof}
    Suppose that the statements $a|b$ and $b|c$ are true. Then there is some integer $m$ such that $am = b$ and there is some integer $n$ such that $bn = c$. By substituting $am = b$ into $bn = c$, we get the following:

    $$
    \begin{aligned}
        c = (am)n\\
        c = a(mn)
    \end{aligned}
    $$

    Since $m$ and $n$ are both integers, then $mn$ must be an integer. Since $a$ times some integer equals $c$, then $a$ divides $c$.\\

    $\therefore$ if $a|b$ and $b|c$, then $a|c$.
\end{proof}

\begin{proposition}{prop3.3:label}
    If $a,b,c \in \Z^+$, then $\lcm(ca,cb) = c\:\lcm(a,b)$
\end{proposition}

\begin{proof}
    Let $a,b,c\in\Z$ and let $d=\:\lcm(a,b)$. Then $a|d$, so for some $l\in\Z$, $d=al$. Then we have $cd = c(al) = (ca)l$, so in particular $ca|cd$. This shows us $cd$ is a common multiple of $ca$ and $cb$. \\

    In fact, any multiple of $ca$ and $cb$ must in particular be divisible by $c$. So it can be written as $ck$ for $k$ a multiple of $a$ and $b$. Then since $c$ is positive, if $d$ is the smallest common multiple of $a$ and $b$, then $cd$ must be the smallest common multiple of $ca$ and $cb$ (since $d<m$ implies that $cd < cm$). \\

    So $\lcm(ca,cb) = c\:\lcm(a,b)$.
\end{proof}

\section{Proof By Cases}

Many times, an effective proof must consider all possible forms that a statement can take on. For example, if your statement in a proof is something along the lines of "$n \in \Z$", there might be different outcomes depending on whether $n$ is positive, negative, or zero. If you can effectively "cover all the bases" and prove each case individually, then you can more effectively make a conslucion about the statement as a whole.\\

When you use cases, ensure that you do the following:
\begin{itemize}
    \item clearly state what the cases are
    \item cover all the possible cases in your proof
\end{itemize}

\begin{proposition}{prop3.4.1:label}
    If $n \in \N$, then $1 + (-1)^n(2n-1)$ is a multiple of 4.
\end{proposition}

\begin{proof}
    Suppose $n \in \N$. Then $n$ is either even or odd. Consider these two cases:\\

    \textbf{Case 1- $n$ is even:} If $n$ is even, then there is some integer  $k$ such that $n = 2k$. Then we can make the following simplifications:
    
    \[
        \begin{aligned}
            1+(-1)^n(2n-1)&=1+(-1)^{2k}(2(2k)-1)\\
            &= 1 + 4k -1 = 4k
        \end{aligned}
    \]

    Therefore, we can conclude that $1+(-1)^n(2n-1)$ is a multiple of 4 if $n$ is even.\\

    \textbf{Case 2- $n$ is odd:} If $n$ is odd, then $n=2k+1$ such that $k\in\Z$.We can simplify in the following way:
    
    \[
    \begin{aligned}
        1+(-1)^n(2n-1) &= 1+(-1)^{2k+1}(2(2k+1)-1)\\
        &= 1 -(4k+2-1)\\
        &=1-(4k+1)=4k
    \end{aligned}
    \]

    Therefore, we can conclude that $1+(-1)^n(2n-1)$ is a multiple of 4 if $n$ is odd.\\

    $\therefore$ if $n \in \N$, then $1 + (-1)^n(2n-1)$ is a multiple of 4. 
\end{proof}

