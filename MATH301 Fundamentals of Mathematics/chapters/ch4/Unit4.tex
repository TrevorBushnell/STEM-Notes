\section{Helpful Definitions For Contrapositive Proofs}

\begin{definition}{def4.1.1:label}
    Suppose that $a,b\in\Z$ and $n\in\N$. We say that $a$ and $b$ are \textbf{congruent modulo $n$} if $n|a-b$. We then write $a\equiv b(\mod n)$.
\end{definition}


\section{What Is the Contrapositive?}

\begin{itemize}
	\item The \textbf{CONTRAPOSITIVE} of $P \implies Q$ is the statement $\sim Q \implies \sim P$.\\

	\item The contrapositive is \textbf{LOGICALLY EQUIVALENT} to the original statement. This means that another way that we can mathematically prove a statement is by trying to prove the contrapositive of the statement rather than the original statement (if the contrapositive is true, then the original statement MUST be true because the contrapositive is logically equivalent to the original statement).\\
	
	\item It is important to note that the contrapositive is \textbf{NOT THE SAME AS THE CONVERSE!}
\end{itemize}

\begin{problem}
	Find the contrapositive, converse, and inverse of the following statement: "If you call your mom, then she will be happy."\\

	\begin{itemize}
		\item \textbf{CONTRAPOSITIVE:} If your mom is not happy, then you did not call her.
		\item \textbf{CONVERSE:} If your mom is happy, then you called your mom.
		\item \textbf{INVERSE:} If you do not call your mom, then your mom will not be happy.
	\end{itemize}
\end{problem}


\newpage
\section{Outline for a Contrapositive Proof}

To prove the statement $P\implies Q$, go through the following outline in your proof:

\begin{itemize}
    \item "Suppose $Q$ is false"
    \item INSERT YOU MATH WORK/LOGIC/EXPLANATIONS
    \item Final step where you end up with $\sim P$
    \item $\therefore$ $\sim P$ is true
    \item $\therefore \sim Q \implies \sim P$ is true
    \item $\therefore$ $P \implies Q$ is true by contradiction
    \item \textbf{End of proof.} 
\end{itemize}


\section{Example Proofs By Contrapositive}

Remember to be careful about negating your $P$ and $Q$ if you are trying to prove a complex statement like this proposition:

\begin{proposition}{prop4.3.3:label}
    Suppose $x,y \in \Z$. If $5 \nmid xy$, then $5 \nmid x$ and $5 \nmid y$.
\end{proposition}

First, we need to note what the contrapositive of the above statement is:

\begin{itemize}
    \item Suppose $x,y \in \Z$. If $5|x$ or $5|y$, then $5|xy$.
\end{itemize}

Now we can engage with the proof (will need to use proof by cases to cover each mini-statement within the "or" statement):

\begin{proof}
    INSERT PROOF HERE
\end{proof}