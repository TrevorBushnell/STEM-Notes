
\documentclass{package/notes}
\usepackage[english]{babel}
\usepackage{amssymb,amsmath,amsfonts}  %%% for maths
%%%%%%%%%%%%%%%%%%%%%%%%%%%%%%%%%%%%%
\usepackage{package/color-env}
\usepackage{lipsum}
\renewcommand\qedsymbol{$\blacksquare$}
\renewcommand{\bf}[1]{\textbf{#1}}
\renewcommand{\it}[1]{\textit{#1}}
%%%%%%%%%%%%%%%%%%%%%%%%%%%%%%%%%%%%%

\begin{document}

	\begin{titlepage} % Suppresses headers and footers on the title page
		
		\centering % Centre everything on the title page
		
		\scshape % Use small caps for all text on the title page
		
		\vspace*{\baselineskip} % White space at the top of the page
		
		%------------------------------------------------
		%	Title
		%------------------------------------------------
		
		\rule{\textwidth}{1.6pt}\vspace*{-\baselineskip}\vspace*{2pt} % Thick horizontal rule
		\rule{\textwidth}{0.4pt} % Thin horizontal rule
		
		\vspace{0.75\baselineskip} % Whitespace above the title
		
		{\huge MATH321: Statistics for Experamentalists\\} % Title
		
		\vspace{0.75\baselineskip} % Whitespace below the title
		
		\rule{\textwidth}{0.4pt}\vspace*{-\baselineskip}\vspace{3.2pt} % Thin horizontal rule
		\rule{\textwidth}{1.6pt} % Thick horizontal rule
		
		\vspace{2\baselineskip} % Whitespace after the title block
		
		%------------------------------------------------
		%	Subtitle
		%------------------------------------------------
		
		
		\vspace*{3\baselineskip} % Whitespace under the subtitle
		
		
		
		\vspace{0.5\baselineskip} 
		
		\vspace{0.5\baselineskip} 
		
		
		\vfill 
		
		%------------------------------------------------
		% Author
		%------------------------------------------------
		
		
		\vspace{0.3\baselineskip} 
		
		
		{\large Edited by\\  Trevor Bushnell} 
		
	\end{titlepage}
	\tableofcontents
%\newpage
\chapter{Unit 1}
\section{Basic Statistical Definitions}

\begin{itemize}
    \item \bf{variable:} characteristic that changes/varies over time and/or for different individuals
    \item \bf{experimental unit:} the individual/object that a variable is being measured on
    \item \bf{measurement:} the result when a variable is actually measured on an experimental unit
    \item \bf{data:} a sample of measurements, which can come from a \it{population} or a \it{sample}.
    \begin{itemize}
        \item \it{population:} set of all measurements of interest
        \item \it{sample:} a subset of the measurements from the population
    \end{itemize}
\end{itemize}

\subsection{Types of Variables}

\begin{itemize}
    \item \bf{qualitative variables (categorical variables):} variables that measure a quality/characteristic
    \item \bf{quantitative variables:} variables that measure a numerical quantity
    \begin{itemize}
        \item \it{discrete variables:} variables that can only have integer values (no decimal points)
        \item \it{continuous variables:} variables that can take on any value (including decimal values)
    \end{itemize}
\end{itemize}


\section{Graphing Qualitative Variables}

\begin{itemize}
    \item We can use a \bf{data distribution} to describe \it{what values} have been measured and \it{how many} times each value has occurred. 
\end{itemize}


\chapter{Unit 2}
\begin{problem}
	LOOK AT SLIDE 2-73\\

	\begin{equation*}
		\begin{aligned}
			&\mu = 150, \sigma = 10\\	
			&\mu \pm \sigma = 150 \pm 10 = 140,160\\
			&\text{68\% of test scores will fall between 140 and 160}\\
			&\mu \pm % COME BACK LATER TO FINISH
		\end{aligned}
	\end{equation*}
\end{problem}

\begin{itemize}
	\item We can approximate $s$ (sample standard deviation) using the range
	\item $s \approx \frac{R}{4}$
	\item Note that $R$ is the range (largest - smallest)
\end{itemize}

\section{Measure of Relative Standing}

\begin{itemize}
	\item How many $\sigma$'s away from the mean does the measurement lie?
	\item We can answer this question using the \bf{z-score}
	\item $z = \frac{x - \bar{x}}{s}$, where $-3 \le z \le 3$
	\begin{itemize}
		\item If you get a z-score outside that range, then that value is an outlier
	\end{itemize}
	\item If you have a set of $n$ measurements on a variable $X$ that are arranged in order of magnitude, then the \bf{p-th percentile} is the value of $X$ that is \it{greater than $p\%$ of measurements and less than the remaining ($(100-p)\%$)}
\end{itemize}

\begin{problem}
	LOOK AT SLIDE 2-73\\

	\begin{equation*}
		\begin{aligned}
			&\mu = 150, \sigma = 10\\	
			&\mu \pm \sigma = 150 \pm 10 = 140,160\\
			&\text{68\% of test scores will fall between 140 and 160}\\
			&\mu \pm % COME BACK LATER TO FINISH
		\end{aligned}
	\end{equation*}
\end{problem}

\begin{itemize}
	\item We can approximate $s$ (sample standard deviation) using the range
	\item $s \approx \frac{R}{4}$
	\item Note that $R$ is the range (largest - smallest)
\end{itemize}

\section{Measure of Relative Standing}

\begin{itemize}
	\item How many $\sigma$'s away from the mean does the measurement lie?
	\item We can answer this question using the \bf{z-score}
	\item $z = \frac{x - \bar{x}}{s}$, where $-3 \le z \le 3$
	\begin{itemize}
		\item If you get a z-score outside that range, then that value is an outlier
	\end{itemize}
	\item If you have a set of $n$ measurements on a variable $X$ that are arranged in order of magnitude, then the \bf{p-th percentile} is the value of $X$ that is \it{greater than $p\%$ of measurements and less than the remaining ($(100-p)\%$)}
\end{itemize}


\end{document}
.