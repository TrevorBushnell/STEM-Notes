\section{Basic Statistical Definitions}

\begin{itemize}
    \item \bf{variable:} characteristic that changes/varies over time and/or for different individuals
    \item \bf{experimental unit:} the individual/object that a variable is being measured on
    \item \bf{measurement:} the result when a variable is actually measured on an experimental unit
    \item \bf{data:} a sample of measurements, which can come from a \it{population} or a \it{sample}.
    \begin{itemize}
        \item \it{population:} set of all measurements of interest
        \item \it{sample:} a subset of the measurements from the population
    \end{itemize}
\end{itemize}

\subsection{Types of Variables}

\begin{itemize}
    \item \bf{qualitative variables (categorical variables):} variables that measure a quality/characteristic
    \item \bf{quantitative variables:} variables that measure a numerical quantity
    \begin{itemize}
        \item \it{discrete variables:} variables that can only have integer values (no decimal points)
        \item \it{continuous variables:} variables that can take on any value (including decimal values)
    \end{itemize}
\end{itemize}


\section{Graphing Qualitative Variables}

\begin{itemize}
    \item We can use a \bf{data distribution} to describe \it{what values} have been measured and \it{how many} times each value has occurred. 
\end{itemize}