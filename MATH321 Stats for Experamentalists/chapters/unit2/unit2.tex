\begin{problem}
	LOOK AT SLIDE 2-73\\

	\begin{equation*}
		\begin{aligned}
			&\mu = 150, \sigma = 10\\	
			&\mu \pm \sigma = 150 \pm 10 = 140,160\\
			&\text{68\% of test scores will fall between 140 and 160}\\
			&\mu \pm % COME BACK LATER TO FINISH
		\end{aligned}
	\end{equation*}
\end{problem}

\begin{itemize}
	\item We can approximate $s$ (sample standard deviation) using the range
	\item $s \approx \frac{R}{4}$
	\item Note that $R$ is the range (largest - smallest)
\end{itemize}

\section{Measure of Relative Standing}

\begin{itemize}
	\item How many $\sigma$'s away from the mean does the measurement lie?
	\item We can answer this question using the \bf{z-score}
	\item $z = \frac{x - \bar{x}}{s}$, where $-3 \le z \le 3$
	\begin{itemize}
		\item If you get a z-score outside that range, then that value is an outlier
	\end{itemize}
	\item If you have a set of $n$ measurements on a variable $X$ that are arranged in order of magnitude, then the \bf{p-th percentile} is the value of $X$ that is \it{greater than $p\%$ of measurements and less than the remaining ($(100-p)\%$)}
\end{itemize}