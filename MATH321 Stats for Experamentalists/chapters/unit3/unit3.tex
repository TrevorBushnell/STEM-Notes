\section{What Is Probability?}

\begin{itemize}
    \item probability is \it{the science of uncertainty}. When we run an experiment, we are unsure of what the outcome will be.
    \item the definition of probability is \bf{the chance of the event occurring}
    \item probability is a numerical value between 0 and 1 (inclusive) describing how likely an event is to occur - the larger the number, the more likely the event will occur.
\end{itemize}


\section{Events and the Sample Space}

\begin{itemize}
    \item \bf{experiment:} the process by which an observation is obtained
    \item a \bf{simple event} is the outcome that is observed on a single repitition of the experiment
    \begin{itemize}
        \item one and only one simple event can occur when the experiement is performed
        \item A simple event is denoted by $E$ with some subscript
        \item each simple event has some probability
    \end{itemize}

    \item \bf{sample space $S$:} the set of all simple events of an experiment
    \begin{itemize}
        \item \it{discrete sample space:} a sample space that contains a finite or countable infinite number of elements
        \item \it{continuous sample space:} a sample space that contains all or a subset of all real numbers
    \end{itemize}

    \item \bf{event:} a collection of one or more simple events (aka a subset of the sample space)
    \begin{itemize}
        \item EX: If the values on a 6-sided die are our sample space, then an example event would be getting an odd number
    \end{itemize}

    \item \bf{mutually exclusive:} a way to describe two events where if one event occurs, the other event cannot occur (same goes in reverse order)
\end{itemize}


\section{Set Theory Review}

\begin{itemize}
    \item Since events are just sets, we can compare relations between events by talking about relations between sets (aka using set theory)
    \item \bf{union $A \cup B$:} the event that occurs when either $A$ or $B$ occur
    \begin{itemize}
        \item $A \cup B = \{x : x \in A $ OR $x \in B\}$
    \end{itemize}
    \item \bf{intersection $A \cap B$:} the event that occurs when both $A$ and $B$ occur
    \begin{itemize}
        \item $A \cap B = \{x : x \in A $ AND $x \in B\}$
    \end{itemize}
    \item \bf{complement $A^C$ or $A'$ or $\bar A$:} means that we care about all the elements in the sample space that are NOT also in $A$. 
    \begin{itemize}
        \item $A^C = \{x : x \notin A\}$
    \end{itemize}
\end{itemize}

\begin{definition}[The Axioms of Probability]{def3.1:label}
    \begin{enumerate}
        \item For any event $A$, $P(A) \ge 0$
        \item $P(S) = 1$
        \item If $A_1, A_2, A_3, \cdots$ is a finite or countably infinite sequence of mutually exclusive events of $S$, then $P(A_1 \cup A_2 \cup A_3) = P(A_1)+P(A_2)+P(A_3)$
    \end{enumerate}
\end{definition}

Using the axioms above, then you can come up with these theorems:

\begin{theorem}[Consequences of the Axioms of Probability]{th3.1:label}
    \begin{enumerate}
        \item If $A$ and $A^C$ are complementary events in a sample space $S$, then $P(A^C) = 1 - P(A)$
        \item $P(\varnothing) = 0$ for any sample space $S$
        \item If $A$ and $B$ are events in a sample space $S$ and $A \subset B$, then $P(A) \le P(B)$
        \item If $A$ and $B$ are any two events in a sample space $S$, then $P(A \cup B) = P(A) + P(B) - P(A \cap B)$
    \end{enumerate}
\end{theorem}


\section{Permutations and Combinations}

\begin{definition}[The Counting Rule (Multiplication Rule)]{def3.4:label}
    If an experiment is performed in two stages with $m$ ways to accomplish the first way and $n$ ways to accomplish the second way, then there are $mn$ ways to accomplish the experiment.\\

    If you have up to $k$ different steps/stages, then you can multiply the number of ways to accomplish each step/stage, so $n_1n_2n_3\cdots n_k$.
\end{definition}

\begin{definition}[Permutations]{def3.5:label}
    The number of ways to arrange $n$ distinct objects where the order is important and you are taking $r$ objects at a time, then this is called the \bf{permutation} and can be computed using:

    $$
        \frac{n!}{(n-r)!} 
    $$
\end{definition}

\begin{definition}[Combinations]{def3.6:label}
    The number of ways to arrange $n$ distinct objects where the order is NOT important and you are taking $r$ objects at a time, then this is called the \bf{permutation} and can be computed using:

    $$
        \frac{n!}{r!(n-r)!} 
    $$
\end{definition}


\section{Calculate Probabilities Using Simple Events}

\begin{definition}[Probability of a Simple Event]{def3.2:label}
    The probability of event $A$ is equal to 

    $$
    \frac{n}{N}
    $$

    where $n = |A|$ and $N = |S|$. In other words, the probability of a simple event is equal to the number of possible outcomes for the desired event divided by the total number of outcomes
\end{definition}

\begin{theorem}[The Law of Large Numbers]{th3.3:label}
    As the number of times an experiment is repeated increases, the relative frequency of a particular outcome will approach the probability of the outcome. \\

    This means that the sample mean $\bar{x}$ will approach the population mean $\mu$.
\end{theorem}

\begin{proposition}[How to Calculate the Probability of an Event]{prop1.1:label}
    \begin{enumerate}
        \item List all the simple events in the sample space $S$
        \item Assign a probability to each simple event
        \item Determine which simple events result in the event of interest
        \item Sum the probabilities of the simple events that result in the event of interest
    \end{enumerate}
\end{proposition}


\section{Calculating }