\section{Definition of Random Variables}

\begin{itemize}
    \item If we have an experiment with a sample space $S$, then a \bf{random variable} $X$ is a set function that assigns unique real numbers $X(s)=x$ to each element $s \in S$
    \begin{itemize}
        \item IN OTHER WORDS: This variable $X$ takes on some value based on the random outcome from the experiment
    \end{itemize}

    \item The \bf{space} of $X$ is the set of all real numbers: $\{x:X(s) = x, s \in S\}$
    \item We typically represent random variables with capital letters while the lowercase letters are your typical variables that you encounter in algebra/calculus that can take on any possible value
\end{itemize}


\section{Discrete VS Continuous Random Variables}

\begin{itemize}
    \item \bf{discrete random variables:} random variables whose set of possible values is either finite or countably finite (can only take on whole number values)
    \item \bf{continuous random variables:} random variables whose set of values is uncountable (can take on whole value and decimal values)
\end{itemize}


\section{Probability Functions}

\begin{definition}[Probability Mass Function (pmf)]{def4.1:label}
    The \bf{probability mass function} is a function that describes a \it{discrete random variable} that satisfies the following properties:

    \begin{enumerate}
        \item $P(X = x) = f(x) > 0$ if $x \in S$
        \item $\sum_{x \in S} f(x) = 1$
        \item $P(A) = \sum_{x \in A} f(x)$ where $A \in S$
    \end{enumerate}
\end{definition}

\begin{definition}[Cumulative Distribution]{def4.2:label}
    If $X$ is a discrete random variable, then the function given by:

    $$
    F(x) = P(X \le x) = \sum_{t \le x} f(t)
    $$

    for $-\infty < x < \infty$ is called the \bf{cumulative distribution function (cdf)}. The cumulative distribution satisfies the following properties:

    \begin{enumerate}
        \item $F(-\infty) = 0$ and $F(\infty) = 1$
        \item if $a < b$, then $F(a) \le F(b)$ for any real numbers $a$ and $b$
    \end{enumerate}

    \begin{itemize}
        \item Random variables are defined in terms of functions which in turn must have distributions.
        \item This means that we can find the mean and the standard deviation
        \item the \it{mean} is also called the \bf{expected value} and is calculated using $\mu = E(x) = \sum xf(x)$
        \item the \it{variance} is calculated using: $\sigma = \sum (x-\mu)^2f(x)$
        \item you can also calculate variance using: $\sigma^2 =E(X)^2 - [E(X)]^2$
        \item The \it{standard deviation} can be calculated using $\sigma = \sqrt{\sigma^2}$
    \end{itemize}
\end{definition}

    \begin{itemize}
        \item The above only works for discrete random variables. For continuous random variables, we can't assign nonzero probabilities to every value or else the function will not sum to 1
        \item This means that for continuous random variables we care about the probability that the value is within a specific \it{range of values}   
        \item A smooth curve defines the probability density function of a continuous random variable 
    \end{itemize}

\begin{definition}[Probability Density Function (pdf) for Continuous Random Variables]{def4.3:label}
    The \bf{probability density function (pdf)} of a continuous random variable is an \it{integrable} function $f(x)$ that satisfies the following properties:

    \begin{enumerate}
        \item $f(x) > 0, x \in S$
        \item $\int_S f(x)\:dx = 1$
        \item If $(a,b) \subseteq S$, then the probability of the event $\{a < X < b\}$ is\\ $P(a < X < b) = \int_a^b f(x)\:dx$
    \end{enumerate}

    Some other important things to note:

    \begin{itemize}
        \item The total area under the curve should be 1
        \item The area of an exact point is always 0 because $a=a, \int_a^a f(x)\:dx = 0$
        \item For continuous random variables, comparison operators can be interchanged and we still get the same probability (NOTE THAT THIS IS NOT TRUE FOR DISCRETE VARIABLES)
    \end{itemize}
\end{definition}

\begin{definition}[Cumulative Distribution Function (cdf)]{def4.4:label}
    A \bf{cumulative distribution function (cdf)} of a random variable $X$ that is continuous is defined in terms of the pdf of $X$ and is given by:

    $$
    F(x) = P(X \le x) = \int_{-\infty^x}f(t)\:dt, -\infty < x < \infty
    $$

    where $F(-\infty) = 0, F(\infty) = 1, F(a) \le F(b)$ when $a<b$.
\end{definition}

\begin{theorem}[Relating pdf and cdf]{th4.5:label}
    If $f(x)$ and $F(x)$ are the values of the probability density and distribution of $X$ and $x$, then:

    $$
    P(a < X < b) = F(b) - F(a)
    $$

    for any constants $a$ and $b$, and:

    $$
    f(x) = \frac{dF(x)}{dx}
    $$

    where the derivative exists.
\end{theorem}


\section{Percentile}

\begin{itemize}
    \item The \bf({(100$p$})th percentile) is a number $\pi_p$ such that the area under $f(x)$ to the left of $\pi_p$ is $p$.
    \item Another way to phrase this: $P(X \le \pi_p) = \int_{-\infty}^{\pi_p}f(x)\:dx = p$
    \item We can use this fact to find the position where this percentile occurs
    \item Recall that Q1 is 25\% percentile, Q3 is 75\% percentile, median is 50\% percentile 
\end{itemize}

\section{Expected Value and Variance of Continuous Random Variables}

If $X$ is a continuous random variable, then:\\

Expected Value:

$$
    \mu = E(X) = \int_{-\infty}^ x\:f(x)\:dx
$$

Variance:
$$
    \sigma^2 = Var(X) = E[(X - \mu)^2] = E[(X - E(X))^2] = E[X^2] - (E[X])^2
$$


\section{Properties of Mean and Variance}

For any discrete or continuous random variable $X$ and constants $a$ and $b$, then\dots

$$
    \begin{aligned}
        E[aX + b] &= aE[x]+b\\
        Var(aX+b) &= a^2\:Var(X)
    \end{aligned}
$$

Similarly, since $E[X] = \mu$ and $Var(X) = \sigma^2$, then for any z-score $z = \frac{X-\mu}{\sigma}$, then $E[z] = 0$ and $Var(z) = 1$.