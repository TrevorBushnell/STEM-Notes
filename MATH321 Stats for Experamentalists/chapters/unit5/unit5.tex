\section{Introduction}

\begin{itemize}
    \item Discrete random variables take on only a finite or countable infinite number of values
    \item There are three discrete probability distributions which will serve as modes for a large number of stats models/applications, which are as follows:
    \begin{enumerate}
        \item \bf{binomial} distribution
        \item \bf{poisson} distribution
        \item \bf{hypergeometric} distribution
    \end{enumerate}
\end{itemize}

\section{Binomial Distribution}

\begin{itemize}
    \item Bernoulli trial is an experiment with only two possible outcomes (success and failure)
    \item These have probabilities $p$ and $1-p$
    \item A coin tossing experiment is an example that uses this distribution 
\end{itemize}

\subsection{The Binomial Experiment}

\begin{enumerate}
    \item Consists of $n$ identical trials
    \item Each trial has one of two outcomes, success or failure
    \item The probability for each trial remains constant
    \item the trials are independent
    \item We are interested in $X$, the number of successes in $n$ trials
\end{enumerate}\newpage

\begin{definition}[The Binomial Prbability Distribution]{def5.1:label}
    For a binomial experiment that has $n$ trials and probability $p$ of a success, then the probability of $k$ successes in $n$ trials is:

    $$
    P(X = k) = \binom{n}{k}p^kq^{n-k} \text{ for } 0 \le k \le n
    $$
\end{definition}