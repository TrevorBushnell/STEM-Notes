\section{Introduction}

\begin{itemize}
    \item Discrete random variables take on only a finite or countable infinite number of values
    \item There are three discrete probability distributions which will serve as modes for a large number of stats models/applications, which are as follows:
    \begin{enumerate}
        \item \bf{binomial} distribution
        \item \bf{poisson} distribution
        \item \bf{hypergeometric} distribution
    \end{enumerate}
\end{itemize}

\section{Binomial Distribution}

\begin{itemize}
    \item Bernoulli trial is an experiment with only two possible outcomes (success and failure)
    \item These have probabilities $p$ and $1-p$
    \item A coin tossing experiment is an example that uses this distribution 
\end{itemize}

\subsection{The Binomial Experiment}

\begin{enumerate}
    \item Consists of $n$ identical trials
    \item Each trial has one of two outcomes, success or failure
    \item The probability for each trial remains constant
    \item the trials are independent
    \item We are interested in $X$, the number of successes in $n$ trials
\end{enumerate}\newpage

\begin{definition}[The Binomial Prbability Distribution]{def5.1:label}
    For a binomial experiment that has $n$ trials and probability $p$ of a success, then the probability of $k$ successes in $n$ trials is:

    $$
    P(X = k) = \binom{n}{k}p^kq^{n-k} \text{ for } 0 \le k \le n
    $$
\end{definition}

Here are some formulas for the basic stats of the Binomial Bistribution:

\begin{itemize}
    \item \bf{MEAN (Expected Value):} $\mu = np$
    \item \bf{VARIANCE:} $\sigma^2 = npq$
    \item \bf{STD. DEV.:} $\sigma = \sqrt{npq}$
\end{itemize}


\section{Poisson Random Variable}

The Poisson random variable is a probability distribution which represents the total number of occurrences of a specified event in a given unit of time/space.

Some examples of a Poisson distribution:

\begin{itemize}
    \item \# of calls received by a switchboard during a given period of time
    \item \# of machine breakdowns in a day
    \item \# of traffic intersection accidencts during a specific time interval
\end{itemize}

\begin{definition}[The Poisson Probability Distribution]{def5.3:label}
    If $X$ is the number of events that occur in a given space/time and $\mu$ is the average number of events that can be expected to occur, then:

    $$
    P(X = k) = \frac{\mu^ke^{-\mu}}{k!}
    $$
\end{definition}

Here are some formulas for the basic stats of the Binomial Bistribution:

\begin{itemize}
    \item \bf{MEAN (Expected Value):} $\mu$
    \item \bf{VARIANCE:} $\sigma^2 = \mu$
    \item \bf{STD. DEV.:} $\sigma = \sqrt{\mu}$
\end{itemize}


\section{Hypergeometric Distribution}

\subsection{The Process/Idea}

\begin{enumerate}
    \item Consists of $n$ trials
    \item Each trial is either a \it{success} or \it{failure}
    \item The probability of success is not the same on each trial, $\therefore$ the events are \bf{dependent}
\end{enumerate}


\begin{definition}[The Hypergeometric Distribution]{def5.4:label}
    If we have $M$ objects and $N-M$ of some other object. If we select $n$ objects and record the number of objects that are of type $M$ (quantify this as $k$), then:

    $$
    P(X = k) = \frac{\binom{M}{k}\binom{N-M}{n-k}}{\binom{N}{n}}
    $$
\end{definition}

Here are some formulas for the basic stats of the Binomial Bistribution:

\begin{itemize}
    \item \bf{MEAN (Expected Value):} $\mu = n \left(\frac{M}{N}\right)$
    \item \bf{VARIANCE:} $\sigma^2 = n\left(\frac{M}{N}\right)\left(\frac{N-M}{N}\right)\left(\frac{N-n}{N-1}\right)$
    \item \bf{STD. DEV.:} $\sigma = \sqrt{n\left(\frac{M}{N}\right)\left(\frac{N-M}{N}\right)\left(\frac{N-n}{N-1}\right)}$
\end{itemize}

\bf{NOTE: Hypergeoemtric is for \it{dependent} events and Binomial is for \it{independent} events}