\section{Continuous Uniform Random Variable}

\begin{itemize}
    \item used to model behavior of a random vairable whose values are uniformily/evenly distributed
    \item pdf for this random variable is $f(x) = \frac{1}{b-a}; \:a \le x \le b$
    \item the graph of this is just simply a rectangle
    \item probability can be calculated as the area under the rectangle over that area
    \item MEAN: $\mu = \frac{b+a}{2}$
    \item VARIANCE: $\sigma^2 = \frac{(b-a)^2}{12}$
\end{itemize}


\section{Exponential Probability Distribution}




\section{Normal Probability Distribution}

Lots of random variables in the real world follow a sort of mound-shaped distribution, which we call a \bf{normal probability distribution}. 

\begin{definition}[The Normal Probability Distribution]{def5.a:label}
    The \bf{Normal probability distribution} follows the following pdf:

    $$
    f(x) = \frac{1}{\sigma\sqrt{2\pi}}e^{\frac{-(x-\mu)^2}{2\sigma^2}}, \:\: -\infty < x < \infty
    $$

    Where $\pi, e$ are the traditional mathematical constants, $\mu$ is the population mean and $\sigma$ is the population standard deviation.
\end{definition}

\begin{itemize}
    \item In a Normal distribution, $\mu$ is the center of the graph and $\sigma$ determines the shape of the graph (smaller $\sigma$ means a more narrow mound-shaped graph, while a larger $\sigma$ means a more spread out/flatter mound-shaped graph)
    \item The distribution is \it{symmetric} about $\mu$ which means half the area is below $\mu$ and half the area is above $\mu$. 
\end{itemize}


\section{Standard Normal Random Variable}

We can \bf{standardize} a value $x$ as the number of standard deviations $\sigma$ it lies from the mean $\mu$. \\

The \bf{standardized Normal random variable} $z$ is defined as:

$$
    \begin{aligned}
        z &= \frac{x-\mu}{\sigma}\\
        x &= \mu + z\sigma
    \end{aligned}
$$

This is the same as the z-score used earlier in the course to determine outliers. We can notice the following patterns

\begin{itemize}
    \item If $x < \mu$, then $z < 0$
    \item If $x > \mu$, then $z < 0$
    \item If $x = \mu$, then $z=0$
\end{itemize}

The area under the standard normal curve to the left of a specific value of $z$ - say $b$ - is the probability $P(z \le b)$.\\

Likewise, the area under the standard normal curve to the right of a specific value of $z$ - say $b$ - is the probability $P(z \ge b) = 1 - P(z < b)$\\

Finally, the area between two possible values of $z$ - say $a$ and $b$ - is probability $P(a \le z \le b) = P(z \le b) - P(z \le a)$\\

You can use Table 3 in the Appendix of these notes to find these probability values (you can also use your calculator to compute these values).


\section{The Normal Approximation to the Binomial Probability Distribution}

You can approximate binomial distributions with a Normal distribution where the mean of your Normal distribution would be the mean of the binomial distribution ($\mu = np$) and the standard deviation of the Normal distribution would be the standard deviation of the binomial distribution ($\sigma = \sqrt{npq}$). This approximation will be adequate \bf{as long as $n$ is sufficiently large and $p$ is not very close to 0 or 1}.\\

You might recall that the area under the bars in a binomial histogram is the same as $P(X=a)$. Because of this, we can approximate this area using the area under the Normal curve over the same region (because we are approximating our binomial distribution with a Normal distribution).\\

The Normal approximation works well when the binomial distribution is roughly symmetric. This will happen when we can spread out at least two standard deviations without exceeding the limits $0$ and $n$. As such, we can check that \bf{the Normal approximation will be adequate if $np > 5$ and $nq > 5$}.

\newpage
\begin{definition}[How to Calculate Binomial Probabilities Using Normal Approximations]{def5.b:label}
    \begin{enumerate}
        \item Find the values of $n$ and $p$
        \item Calculate $\mu = np$ and $\sigma = \sqrt{npq}$
        \item Write out the probability you need in terms of $x$ to find the area under the curve
        \item Correct the value of $x$ by $\pm 0.5$ to include the continuity correction
        \item Convert the $x$ values to $z$ values using $\frac{x \pm 0.5 - np}{\sqrt{npq}}$
        \item Use Table 3 to calculate the approximate probability
    \end{enumerate}

    You can use these following continuity corrections:

    $$
    \begin{aligned}
        a \le x \le b &\implies (a-0.5) \le x \le (b+0.5)\\
        x = a & \implies a-0.5 \le x \le a + 0.5
    \end{aligned}
    $$
\end{definition}

