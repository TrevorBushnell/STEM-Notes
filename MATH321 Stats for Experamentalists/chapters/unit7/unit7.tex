\section{Introduction}

Recall parameter VS statistic:

\begin{itemize}
    \item \bf{parameter:} numerical descriptive measures for \it{populations}
    \item \bf{statistic:} numerical descriptive measures for \it{samples}
\end{itemize}

We are going to to take SAMPLES so that we can make inferences about the ENTIRE POPULATION.

There are multiple ways to create a sample:

\begin{itemize}
    \item \bf{simple random sample:} randomly choose $n$ people with equal probability from your entire population
\end{itemize}


\subsection{Statistics}

\begin{itemize}
    \item When we select a random sample from a population, the numerical descriptives that we calculate from that sample are called \it{statistics}
    \item Because these statistics will change depending on the sample that we take, each statistic is actually a \it{random variable}
\end{itemize}


\subsection{Ways of Taking Our Sample}

\begin{itemize}
    \item \bf{without replacement:} the total number of possible samples are $\binom{N}{n}$, and the probability is $\frac{1}{\binom{N}{n}}$
    \item \item \bf{with replacement:} the total number of possible samples are $N^n$, and the probability is $\frac{1}{N^n}$
\end{itemize}