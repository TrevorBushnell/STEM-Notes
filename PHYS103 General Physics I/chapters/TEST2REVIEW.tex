\chapter{TEST \#2 REVIEW}

\section{Friction}

\begin{itemize}
    \item The static friction force and the applied force are going to be equal until a certain breakaway point
    \item Once we pass the “friction breaking point”, then we get kinetic friction which becomes a CONSTANT force
    \item Because there are two types of friction forces, DON’T SIMPLY SAY $F_f = \mu F_n$!!!
\end{itemize}


\section{Centripetal Forces}

\begin{itemize}
    \item Centripetal forces are forces that cause an object to continue moving in a circle
    \item this means that Newton’s Second Law becomes $F=ma_c$
    \item Centripetal forces always point towards the CENTER of the circle
    \begin{itemize}
        \item This means that at the top of the circle, the centripetal force points down
        \item at the bottom of the circle, the centripetal force points up
    \end{itemize}
\end{itemize}


\section{Work}

\begin{itemize}
    \item $W = F \cdot d$
    \item SI Units: Joules
    \item If the applied force is not constant, then you know that $W = \int F(x)dx$
    \begin{itemize}
        \item Note that you may not have to calculate an integral, but you should know that if you have a graph of Force as a function of distance and you are being asked to find work, then you should compute the \textbf{area under the curve}
    \end{itemize}
\end{itemize}


\section{Energy}

\begin{itemize}
    \item If work is being done on an object, then the object has some energy
    \item \textbf{Kinetic Energy:} energy if the object is in motion: $KE = \FRAC{1}{2}mv^2$
    \item \textbf{Gravitational Potential Energy:} energy if the object is hovering in the air: $PE = mgh$
    \item \textbf{Spring Energy:} energy if the object compresses the spring
    \item \textbf{Thermal Energy:} Energy that is lost to heat to the environment
    \item $W = \Delta KE + \Delta PE + \Delta E_s + E_{TH}$
    \item If no external forces are applied on the system, then the net work of the system is ZERO
\end{itemize}

\section{Test Question Examples}

\begin{problem}
    A block of mass $m$ is pushed off a spring with spring constant $K$ which was compressed a distance $\Delta x$. The block moves up a slope that is $H$ meters high, then returns to ground level and slides across a source of friction with coefficient of kinetic friction $\mu_k$. \\
    
    a) What is the speed of the block just after the block starts moving after the spring initially pushes the block off?
    
    $$
    \begin{aligned}
        W &= \Delta KE + \Delta PE + \Delta E_s + E_{TH}\\
        0 &= \frac{1}{2}mv_A^2 + \frac{1}{2}K(-\Delta x^2)\\
        v_A &= \sqrt{\frac{K}{m}\Delta x^2}
    \end{aligned}
    $$
    
    b) What is the speed of the block at the top of the slope?
    
    $$
    \begin{aligned}
        W &= \Delta KE + \Delta PE + \E_s + E_{TH}\\
        0 &= \frac{1}{2}mv_B^2+mgH + \frac{1}{2}K(-\Delta x^2)\\
        v_B &= \sqrt{\frac{K\Delta x^2 - mgH}{m}}
    \end{aligned}
    $$
    
    c) How far does the block slide on the friction surface before the block stops?
    
    $$
    \begin{aligned}
        W &= \Delta KE + \Delta PE + \Delta E_s + E_{TH}\\
        0 &= \frac{1}{2}m(-v_B^2) + mg(-H) + F_fL\\
        0 &= \frac{1}{2}m(-v_B^2) + mg(-H) + \mu_kmgL\\
        L &= \frac{\frac{1}{2}mv_B^2+gH}{\mu_kg}
    \end{aligned}
    $$
\end{problem}

\section{WHAT IS ON THE EXAM TOMORROW}

\begin{itemize}
    \item Spud Gun
    \item Car
    \item Siding Block
    \item Graph
    \item Ramp/Spring/Box
    \item BONUS: Quadratic Equation
\end{itemize}