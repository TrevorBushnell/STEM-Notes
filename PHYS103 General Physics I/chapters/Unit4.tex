\section{Introduction to Work}

\begin{definition}{def4.1:label}
    \textbf{WORK: How much energy it takes to do a certain physical action.}

    $$
    W = \Delta E = \vec F \cdot \vec {\Delta x}
    $$

    $W$ = Work done (SI Units: $\J = \N\m$)\\
    $\Delta E$ = Change in energy\\
    $\vec F$ = the force applied\\
    $\Delta x$ = the displacement over which the object was applied the given force $\vec F$
\end{definition}


\begin{problem}
    A couch is pushed with a force 25 $\N$ over a distance of 5 $\m$. Calculate the work that is applied to the couch. 

    $$
    \begin{aligned}
        W &= F_A \cdot \Delta x\\
        W &= (25 \N)(5 \m)\\
        W &= 100 \J
    \end{aligned}
    $$
\end{problem}


\begin{problem}
    \textbf{SEE ATTACHED FIGURE}

    a) How much energy is expended by the applied force to move the couch 5 $\m$?
    b) How much energy does the frictional force expend to move the couch 5 $\m$?

    To solve part a):
    $$
    \begin{aligned}
        W_A &= F_{Ax} \cdot \Delta x\\
        W_A &= F\cos\theta \cdot \Delta x\\
        W_A &= (50 \N)\cos(25^\circ)(5 \m)\\
        W_A &= 227 \J
    \end{aligned}
    $$

    To solve part b):
    $$
    \begin{aligned}
        W_f &= -F_f \cdot \Delta x\\
        W_f &= -\mu_kF_N \cdot \Delta x\\
        W_f &= -\mu_k(mg + F_A\sin\theta) \cdot \Delta x\\
        W_f &= -(0.15)((20\kg)(9.81\frac{\m}{\s^2}) + (50\N)\sin(25^\circ))
        W_f &= VALUE
    \end{aligned}
    $$
\end{problem}


\section{Energy}

Energy comes in many different forms. The main forms of energy (as well as the proofs to get their respective equations) are listed below:


\begin{definition}[Kinetic Energy]{def4.2:label}
    If an object is moving, the energy that the object expends is equal to the \textbf{kinetic energy}. 

    $$
    KE = \frac{1}{2}mv^2
    $$

    $KE$ = the kinetic energy expended\\
    $m$ = mass of the object\\
    $v$ = velocity of the object at the given moment where you wish to find the energy
\end{definition}

\begin{proof}
    PROOF OF KINETIC ENERGY HERE
\end{proof}


\begin{definition}[Potential Energy]{def4.3:label}
    $$
    PE = mgh
    $$
\end{definition}

\begin{proof}
    PROOF OF POTENTIAL ENERGY HERE
\end{proof}

\begin{definition}[Spring Energy]{def4.4:label}
    $$
    SE = \frac{1}{2}kx^2
    $$
\end{definition}

\begin{proof}
    PROOF OF SPRING ENERGY HERE
\end{proof}


\subsection{The Usefulness of Energy and Work}

While we have an equation for the work if we know the applied forces, if we know other elements about the system (how fast )